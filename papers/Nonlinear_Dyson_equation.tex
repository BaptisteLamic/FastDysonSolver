\documentclass[aps,prl,reprint,amsmath,amssymb]{paper}
\usepackage{amsmath}
\usepackage{amsfonts}
\usepackage{amssymb}
\usepackage{dsfont}
\usepackage{cleveref}
\usepackage{graphicx}
\usepackage{booktabs}
\usepackage{url}

\begin{document}
	
	\title{Solving the Nonlinear Dyson Equation}
	
	\author{Baptiste Lamic}
	
	\date{\today}
	
	\maketitle
	
	\section{Problem}
	We solve the following Dyson-like equation:
	\begin{equation}
		G_i = g_i + g_i \Delta G_i, \quad \text{for } i \in \{1,2\},
	\end{equation}
	where the \emph{hybridization function} $\Delta$ is defined as:
	\begin{equation}
		\Delta(t,t') = J \frac{G_1(t,t') + G_2(t,t')}{2}.
	\end{equation}
	The \emph{free Green functions} correspond to the non-interacting Hamiltonians $h_1(t) = U(t)/2$ and $h_2(t) = -U(t)/2$, where $U(t)$ is a given function. The retarded free Green functions satisfy:
	\begin{equation}
		\frac{\text{d} g^R_i(t,t')}{\text{d} t} + i h_i(t) g^R_i(t,t') = -i \delta(t-t'). \label{eq:ode_free_Green_function}
	\end{equation}
	
	In the real-time Keldysh formalism, we solve the following equations:
	\begin{align}
		G^\text{R} &= g^\text{R} + g^\text{R} \Delta^\text{R} G^\text{R}, \label{eq:retarded_dyson_equation} \\
		G^\text{A} &= {G^\text{R}}^\dagger, \\
		G^\text{K} &= (\mathds{1} + G^\text{R} \Delta^\text{R}) g^\text{K} (\mathds{1} + \Delta^\text{A} G^\text{A}) + G^\text{R} \Delta^\text{K} G^\text{A}, \label{eq:kinetic_dyson}
	\end{align}
	where
	\begin{align}
		\Delta^\text{R}(t,t') &= \frac{G_1^\text{R}(t,t') + G_2^\text{R}(t,t')}{2}, \\
		\Delta^\text{K}(t,t') &= \frac{G_1^\text{K}(t,t') + G_2^\text{K}(t,t')}{2}.
	\end{align}
	
	\section{Algorithm}
	
	\subsection{Discretization}
	The kernels are discretized on a uniform time grid spanning from \( t = 0 \) to \( t_f \):
	\begin{equation}
		\mathbf{A}[p,q] \equiv A(t_p,t_q), \qquad \text{where} \quad t_p = p\delta_t.
	\end{equation}
	The number of time steps is given by \( N \equiv t_f / \delta_t \).
	In the following, we consider the normalized Frobenius norm:
	\begin{equation}
		\| \cdot \| : A \mapsto \frac{1}{N} \sqrt{\sum_{p,q} \left(\mathbf{A}[p,q]\right)^2}.
	\end{equation}
	This norm can be efficiently evaluated using matrix operations on the compressed representation, leveraging the relation:
	\begin{equation}
		\| \mathbf{A} \| = \sqrt{\text{tr}(\mathbf{A}^\dagger \mathbf{A})} / N .
	\end{equation}

	
	Integral operator are discritized by the Nystrom method using a trapezoidale quadrature. 
	
	\subsection{Construction of $g_i$ and $J$}
	The operator $J$ is diagonal, making its construction trivial.
	To compress the free Green functions efficient implementation of the following maps are required
	\begin{align}
		\mathbf{u} &\mapsto \sum_q g_i(t_p, t_q) u_q, \\
		\mathbf{u} &\mapsto \sum_q g_i^\dagger(t_p, t_q) u_q, \\
		(p,q) &\mapsto g_i(t_p, t_q).
	\end{align}
	
	 From \Cref{eq:ode_free_Green_function}, we deduce:
	\begin{equation}
		g_i^R(t,t') = \Theta(t-t') g_i^R(t,0) g_i^R(t',0)^{-1}, \label{eq:gr_naive_factorisation}
	\end{equation}
	where $\Theta$ is the Heaviside function. Hence, the maps can be realized by solving $g_i^R(t,0)$ with a classical ODE solver and leveraging the low-rank structure of $g_i^R$.
	
	\subsubsection{Dealing with relaxation in  $g^R$}
	Even in presence of relaxation, one expects $\| g^R(t>0, 0) \| > 0$ . Yet, simple relaxation models lead to $\log\left(\| g^R(t>0, 0) \| \right) \propto -t$. Hence, when doing the computation in finite precision arithmetic \cref{eq:gr_naive_factorisation} quickly become useless. A very direct approach to bypass this issue would be to discretize the differential operator directly, and then inverse it.  
	
	\subsection{Parmeters}
	We set :
	\begin{equation}
		U(t) = \alpha \sin(2 t) \exp(-t/\tau) (1-\exp(-t/4\tau))
	\end{equation}
	\begin{equation}
		J = 1
	\end{equation}
	
	\subsection{Iteration Method}
	We iteratively compute:
	\begin{align}
		G_{i,j+1} &= g_i + g_i \Delta_j G_{i,j+1}, \\
		\Delta_j &= \frac{G_{1,j} + G_{2,j}}{2}.
	\end{align}
	
	\subsection{Convergence Monitoring}
	We define the norm of the discretized kernel $A(t_p, t_q)$ with $t_{i,p}$
	Convergence is monitored using the following error measures:
	\begin{align}
		\mathcal{E}_{j,\text{abs}} &\equiv \sqrt{\| G^R_{1,j+1} - G^R_{1,j} \|^2 + \| G^R_{2,j+1} - G^R_{2,j} \|^2}, \\
		\mathcal{E}_{j,\text{rel}} &\equiv \frac{ \mathcal{E}_{j,\text{abs}}}{\sqrt{\| G^R_{1,j+1} \|^2 + \| G^R_{2,j+1} \|^2}},
	\end{align}
	
	\subsection{Stopping Criterion}
	The iteration stops when:
	\begin{itemize}
		\item The number of iterations exceeds the preset maximum.
		\item $\mathcal{E}_{j,\text{rel}}$ plateaus.
		\item The target accuracy is reached
	\end{itemize}
	
	\section{Results}
	\begin{figure}[h]
		\centering
		\includegraphics[width=0.9\columnwidth]{../plots/FK_benchmark_hss_vs_full.pdf}
		\caption{
			Convergence toward self-consistent solution for different propagation times $t_f$. 
			The compression accuracy is maintained at its nominal value throughout the computation. 
			\emph{a:} Real and imaginary part of $g^R_1$ 
			\emph{b:} Real and imaginary part of $g^R_2$ 
			\emph{c:} Evolution of relative difference between successive iterations. 
			\emph{d:} Number of iterations required to reach desired accuracy. 
			\emph{Parameters:} $\alpha = 20, \tau = 5$ and $\delta_t = 0.1$. 
			Compression: $rtol_{\text{nominal}} = 10^{-8}$ and $atol_{\text{nominal}} = 10^{-9}$
		}
	\end{figure}
	
	
	\newpage
	\bibliographystyle{apsrev4-1}
	\bibliography{biblio}
	
\end{document}
